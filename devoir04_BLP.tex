\documentclass{beamer}
\usepackage{biblatex}
\usepackage{standalone}
\usepackage{tikz}
\usepackage{tkz-euclide}
\usetikzlibrary{calc,decorations.markings}
\usetheme{Goettingen}
\usecolortheme{default}
\setbeamertemplate{footline}[frame number]

\addbibresource{bibliographie.bib}

\title{Planification de recherche}
\subtitle{Capteur de distance à reconnaissance d'image au laser}
\author{Benjamin Lapointe-Pinel}
\institute{UQAR}
\date{2021-10-31}

\begin{document}

\frame{\titlepage}

\frame{{Table des matières}\tableofcontents}

\section{Formation}
\begin{frame}{Auto apprentissage}
	\begin{itemize}
		\item Physique optique
		\item Systèmes optiques
		\item Conception circuits électriques
		\item Triangulation laser active
		\item Sera fait au fur et à mesure que le besoin se manifeste
		\item Total~: 1 session (4 mois), étalé sur 1 an
	\end{itemize}
\end{frame}

\section{Revue littérature}
\begin{frame}{Revue littérature}
	\begin{itemize}
		\item Inclut un peu l'auto apprentissage
		\item Peut être fait en parallèle avec d'autres activités
		\item Surtout intensive au début
		\item Total~: 1 session (4 mois)
	\end{itemize}
\end{frame}

\section{Génération de données}
\begin{frame}{Génération de données}
	\begin{itemize}
		\item Fabrication de la plateforme de génération de jeux de données~: 1 session (4 mois)
		\item La génération de données prend environ 1 semaine par lot
		\item Plusieurs matériaux et systèmes optiques seront testés
		\item Heureusement, la génération est asynchrone et non bloquante
		\item Il est possible de débuter la recherche avec seulement quelques jeux de données
	\end{itemize}
\end{frame}

\section{Écriture d'articles}
\begin{frame}{Écriture d'articles}
	\begin{itemize}
		\item Minimum un article en perspective, peut-être plus d'un
		\item Inclut l'interprétation de données volumineuses
		\item Je manque d'expérience pour estimer combien de temps prend la diffusion des résultats. J'imagine que ce peut être fait de façon relativement asynchrone.
		\item Total~: 1 session (4 mois) par article
	\end{itemize}
\end{frame}

\section{Rédaction de mémoire}
\begin{frame}{Rédaction de mémoire}
	\begin{itemize}
		\item Inclut la proposition de recherche
		\item Le contenu d'articles peut facilement être réutilisé
		\item Total~: 1 session (4 mois)
	\end{itemize}
\end{frame}

\section{Coûts et matériaux}
\begin{frame}{Coûts et matériaux}
	\begin{itemize}
		\item Livres et articles (coût minime)
		\item Salaire de recherche (environ 10k\$ par année)
		\item Plateforme (quelques milliers de dollars)
			\begin{itemize}
				\item Matériaux
				\item Matériel optique
				\item Matériel informatique
				\item Main d'œuvre
			\end{itemize}
		\item Total~: environ 50k\$
	\end{itemize}
\end{frame}

\section{Conclusion}
\begin{frame}{Conclusion}
	Au final, cela donne une maîtrise d'environ 2 ans, sans considérer les imprévus.
	\\~\\
	Ma maîtrise est à temps partiel, cela prendra donc plus de temps que ce le total porte à croire. De plus, mon horaire de travail et d'étude est sujet à changement. Heureusement, ma date de remise finale tombe à l'automne 2025.
\end{frame}


\end{document}
